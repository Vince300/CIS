\newpage

\chapter{Documentation des outils et technologies utilisés}

\section{Front-end : Serveur nginx + Passenger + Sinatra}

Le front-end repose sur un serveur nginx qui utilise passenger pour executer les scripts ruby (framework Sinatra).
nginx permet de filtrer les requêtes en fonction des certificats et de leur provenance, et les redirige vers le controlleur Sinatra associé à la requête.

\section{Front-end : Lancement de jobs pour les utilisateurs locaux} 

L'administrateur possède un script python {\tt createuser} lui permettant de créer un nouvel utilisateur, ainsi que de l'équivalent {\tt deleteuser} pour supprimer un utilisateur. \\

Chaque utilisateur a accès à un script python {\tt lancerjob} paramétrable, lui permettant de lancer l'exécution d'un job :\\
\begin{description}
	\item[--job] : chemin du job à exécuter
	\item[--in] (optionnel) : chemin d'un dossier de données à envoyer avec le job
	\item[--externalize] (optionnel) : url où envoyer le job. Réparti automatiquement si non spécifié
\end{description}

\section{Workers : Exécution de jobs isolés via Docker}

Sur les ``workers'' qui réalisent les jobs des utilisateurs, Docker a été utilisé pour fournir des environnements
isolés. Les limites d'espace disque, de mémoire et de CPU sont définies lors du lancement d'un conteneur tel qu'indiqué
dans la documentation de Docker\footnote{https://docs.docker.com/engine/reference/run/}.

Docker envoie ensuite des évènements qui indiquent l'état des conteneurs, ce qui permet de détecter la terminaison d'un
job sans utiliser de mécanisme de polling.

Afin d'écouter les évènements Docker et de lancer les jobs à partir des fichiers envoyés par le frontend, un daemon
systemd est exécuté sur les workers. Ce daemon est responsable d'envoyer les fichiers de résultat mais aussi de gérer
l'expiration des conteneurs conformément à la configuration.

Les processus Ruby gérés par nginx/Passenger communiquent avec ce daemon en utilisant dRuby (framework d'appel de
méthode à distance, utilisable isolé sur une machine sur l'interface loopback pour localhost).
