\newpage

\chapter{Documentation d'installation}

\section{Installation des OS}

Pour les trois machines le choix des operating systems s'est porté sur une distribution debian minimaliste (serveur ssh et utilitaires courants). L'installation est manuelle, et les disques sont chiffrés à l'aide de LVM.

\section{Provisionnement}

Pour l'installation de notre architecture, un dossier {\it provisioning} comprend l'ensemble des scripts d'installation,
à exécuter dans l'ordre selon les instructions d'installation (cf. {\it PROVISIONING.md}). Il faut également positioner
les clés et les certificats à la main sur les machines (cf. étape {\tt 04-install-keys-*.txt}).

Ces étapes sont à effectuer après l'installation de l'OS choisi (distribution Debian 8 x64 minimale) et la
configuration de son interface réseau pour qu'il soit accessible depuis le frontend (sur 192.168.0.0/24).

Le dossier {\it control} contient un exécutable {\it control} qui permet d'exécuter certaines tâches de maintenance sur
les différentes classes de machines qui font partie de notre architecture, comme dans les exemples ci-dessous. 

\begin{verbatim}
# Déploiement du code applicatif sur toutes les machines
./control all deploy

# Déploiement du code applicatif sur tous les workers
./control workers deploy

# Déploiement du code applicatif sur le frontend
./control frontend deploy

# Redémarrage des serveurs
./control all restart-services

# Récupération des logs
./control all fetch-logs
\end{verbatim}

Une configuration SSH spécifique ({\it provisioning/config }) permet à l'administrateur de se connecter directement aux
différents workers en passant par un tunnel au niveau du frontend.
